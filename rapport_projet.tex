\documentclass[a4paper,12pt]{article}
\usepackage[utf8]{inputenc}
\usepackage[T1]{fontenc}
\usepackage[french]{babel}
\usepackage{graphicx}
\usepackage{hyperref}
\usepackage{listings}
\usepackage{xcolor}
\usepackage{geometry}
\usepackage{fancyhdr}

% Configuration de la page
\geometry{hmargin=2.5cm,vmargin=2.5cm}
\pagestyle{fancy}
\fancyhf{}
\rhead{Rapport de Projet : EduManager}
\lhead{Laravel CRUD}
\rfoot{Page \thepage}

% Configuration du code
\definecolor{codegray}{rgb}{0.5,0.5,0.5}
\definecolor{codepurple}{rgb}{0.58,0,0.82}
\definecolor{backcolour}{rgb}{0.95,0.95,0.92}

\lstdefinestyle{mystyle}{
    backgroundcolor=\color{backcolour},   
    commentstyle=\color{codegray},
    keywordstyle=\color{magenta},
    numberstyle=\tiny\color{codegray},
    stringstyle=\color{codepurple},
    basicstyle=\ttfamily\footnotesize,
    breakatwhitespace=false,         
    breaklines=true,                 
    captionpos=b,                    
    keepspaces=true,                 
    numbers=left,                    
    numbersep=5pt,                  
    showspaces=false,                
    showstringspaces=false,
    showtabs=false,                  
    tabsize=2
}
\lstset{style=mystyle}

\title{
    \vspace{2cm}
    \textbf{\Huge Application de Gestion des Étudiants}\\
    \vspace{0.5cm}
    \Large Rapport de Réalisation Technique -- EduManager\\
    \vspace{2cm}
    \includegraphics[width=0.4\textwidth]{example-image} % Remplacer par le logo si disponible
    \vspace{2cm}
}
\author{\textbf{Réalisé par :} [Votre Nom]}
\date{\today}

\begin{document}

\maketitle
\thispagestyle{empty}
\newpage

\tableofcontents
\newpage

\section{Introduction}

\subsection{Contexte du Projet}
Dans le cadre de notre formation en développement web, nous avons conçu et réalisé une application complète de gestion des étudiants (EduManager). Ce projet a pour but de mettre en pratique l'architecture \textbf{MVC} (Modèle-Vue-Contrôleur) via le framework \textbf{Laravel 11}.

\subsection{Objectifs}
\begin{itemize}
    \item Maîtriser l'installation et la configuration de l'environnement Laravel.
    \item Implémenter un système d'authentification robuste.
    \item Créer un CRUD (Create, Read, Update, Delete) complet pour les profils étudiants.
    \item Concevoir une interface utilisateur moderne et responsive.
\end{itemize}

\section{Architecture Technique}

\subsection{Technologies Utilisées}
\begin{description}
    \item[Backend :] PHP 8.2, Laravel 11
    \item[Base de données :] MySQL (via XAMPP)
    \item[Frontend :] Blade, CSS3 (Variables CSS), JavaScript
    \item[Outils :] Composer, Artisan CLI, Lucide Icons
\end{description}

\subsection{Modélisation des Données}
L'application repose sur deux entités principales liées par une relation \textit{One-to-One} :
\begin{itemize}
    \item \textbf{User} : Gère les identifiants de connexion et le rôle (admin ou étudiant).
    \item \textbf{Etudiant} : Contient les informations académiques (CNE/Apogée, filière, photo...).
\end{itemize}

\section{Étapes de Réalisation}

\subsection{1. Initialisation du Projet}
Nous avons utilisé Composer pour créer le squelette de l'application :
\begin{lstlisting}[language=bash]
composer create-project laravel/laravel crud_etudiant
cd crud_etudiant
php artisan storage:link # Pour les images publiques
\end{lstlisting}

\subsection{2. Migrations de Base de Données}
La structure de la table \texttt{etudiants} a été définie comme suit dans la migration :

\begin{lstlisting}[language=PHP]
Schema::create('etudiants', function (Blueprint $table) {
    $table->id();
    $table->string('numero_apogee')->unique();
    $table->string('nom');
    $table->string('prenom');
    $table->string('email')->unique();
    $table->string('telephone');
    $table->string('photo')->nullable(); // Gestion de l'upload
    $table->foreignId('user_id')->constrained()->onDelete('cascade');
    $table->timestamps();
});
\end{lstlisting}

\subsection{3. Authentification et Rôles}
Nous avons implémenté un système d'inscription personnalisé dans \texttt{AuthController.php} qui crée simultanément l'utilisateur et son profil étudiant. L'accès à la partie administration est protégé par un Middleware \texttt{AdminMiddleware}.

\subsection{4. Fonctionnalités CRUD}
Le contrôleur \texttt{EtudiantController} gère la logique métier :
\begin{itemize}
    \item \textbf{Index} : Liste paginée avec recherche dynamique.
    \item \textbf{Show} : Page de profil détaillée.
    \item \textbf{Create/Edit} : Formulaires avec prévisualisation de l'image en JavaScript.
    \item \textbf{Destroy} : Suppression sécurisée avec modal de confirmation.
\end{itemize}

\section{Interface Utilisateur (UI/UX)}
Une attention particulière a été portée au design pour offrir une expérience ``Premium'' :
\begin{itemize}
    \item Utilisation d'une palette de couleurs cohérente (Indigo/Violet) via des variables CSS.
    \item Navigation par barre latérale (Sidebar) fixe.
    \item Remplacement des émojis par la bibliothèque d'icônes \textbf{Lucide}.
    \item Feedback utilisateur via des messages de session (Succès/Erreur).
\end{itemize}

\section{Sécurité et Améliorations}
\subsection{Intégration Firebase (Concept)}
Pour la fonctionnalité de ``Mot de passe oublié'', nous avons étudié l'intégration de \textbf{Firebase Authentication}. Cela permet de déléguer l'envoi sécurisé des emails de réinitialisation à l'infrastructure de Google, évitant la configuration complexe d'un serveur SMTP local.

\section{Conclusion}
Ce projet a permis de consolider nos connaissances en développement Fullstack avec PHP/Laravel. L'application est fonctionnelle, sécurisée et présente une interface soignée adaptée aux besoins d'une administration scolaire.

\end{document}
